% Template APA básico (márgenes 2.5 cm, interlineado 1.5, paginación abajo izquierda)
\documentclass[12pt]{article}
\usepackage[spanish]{babel}
\usepackage[utf8]{inputenc}
\usepackage[T1]{fontenc}
\usepackage{setspace}
\usepackage[a4paper,margin=2.5cm]{geometry}
\usepackage{lmodern}
\usepackage{fancyhdr}
\usepackage{tocloft}

% Interlineado 1.5
\onehalfspacing

% Encabezado/pie
\pagestyle{fancy}
\fancyhf{}
% Número de página en el pie, alineado a la izquierda
\fancyfoot[L]{\thepage}

% Portada (sin número)
\begin{document}
\pagenumbering{gobble}

% ----------------- PORTADA -----------------
\begin{center}
    {\Large Agenda Escolar}\\[1.5em]
    {\large Documento en Formato APA}\\[8em]
    {\normalsize Autor: Equipo de Desarrollo}\\[0.5em]
    {\normalsize Fecha: \\ \today}
\end{center}
\newpage

% ----------------- ÍNDICE (sin número) -----------------
\pagenumbering{gobble}
\tableofcontents
\newpage

% ----------------- CONTENIDO (numera desde 1) -----------------
\pagenumbering{arabic}
\setcounter{page}{1}

\section{Resumen e Introducción}
Agenda Escolar es una aplicación web para gestionar tareas, eventos recurrentes, calendario, pagos, exportaciones (Excel/PDF) y notificaciones. Las ocurrencias recurrentes de eventos se expanden bajo demanda según el rango visible del calendario sin materialización en base de datos.

\section{Requisitos Mínimos}
\begin{itemize}
    \item PHP $\geq$ 8.1; Composer; Node.js y npm; SQLite/MySQL.
    \item CPU x64, 4 GB RAM, 500 MB almacenamiento.
\end{itemize}

\section{Framework y Librerías}
Laravel (12.x), Blade, Bootstrap 5, Vite, Vue 3, FullCalendar 6, Laravel Excel (PhpSpreadsheet), DomPDF.

\section{Funciones Principales}
Tareas (CRUD y exportación), eventos con recurrencia on-the-fly, calendario con filtros y estadísticas, pagos con eventos de calendario, notificaciones en canal database.

\section{Instalación y Despliegue}
Ver guía en README\_APA.md. En producción, compilar assets con Vite, configurar APP\_URL y usar MySQL/MariaDB.

\section{Puntos a Mejorar y Visión}
Roles y permisos avanzados, API pública, recordatorios, i18n/acc, reportes analíticos, CI/CD.

\end{document}
